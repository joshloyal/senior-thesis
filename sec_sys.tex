\section{Systematic Uncertainties}
\label{sec:sys}
A complete model of the signal and background distributions not only require
a description of the signal and background shapes observed in data, but also 
a quantification of the various uncertainties involved in the measurement.
The theoretical and experimental systematic uncertainties are summarized in
this section.

\subsection{Theoretical uncertainties}
There are two sources of theoretical uncertainties on the production cross-section
of \HToZg events: the uncertainty related to the energy scales used for the
fixed-order calculation, and the uncertainty from the parton distribution functions
(PDFs) and the value of $\alphas$ used in the perturbative calculation. 

\begin{itemize}
\item \textbf{Scale Uncertainties:} 
In theoretical calculations, the value of observables are obtained using a 
perturbative expansion. This introduces an uncertainty due to missing higher order
corrections. In particular, the \HToZg signal cross-section is typically calculated
with leading-order (LO) matrix elements that can be corrected with higher-order
QCD (the production of a Higgs boson via gluon fusion is a QCD process) and 
electroweak corrections (Vector Boson Fusion production).

\item \textbf{Proton Structure Uncertainties:}
At the LHC Higgs bosons are produced using proton-proton collisions, so
an understanding of the \HToZg cross-section depends on the internal
parton structure of the proton. This internal proton structure cannot be
extracted from perturbation theory ($\alphas$ is large at the low binding
energies of the proton) and must be measured in the form of parton distribution
functions (PDFs). Consequently, the statistical and systematic uncertainties 
of these measurements must be propagated to any calculated cross-section.
\end{itemize}

The Higgs boson production
cross-sections and decay branching fractions as functions of the Higgs boson
mass are compiled, together with uncertainties, in Ref.~\cite{LHCHiggsCrossSectionWorkingGroup:2011ti}.
For each tested Higgs boson mass hypothesis the uncertainties from Ref.~\cite{LHCHiggsCrossSectionWorkingGroup:2011ti} are used.
\refT{tab:theory_uncertainties} summarizes uncertainties for a Higgs boson mass of
125 GeV. They depend only mildly on $m_H$, for $120 \GeV < m_H < 150 \GeV$,
with the exception of the relative uncertainty on the \HToZg branching fraction,
which varies between 9.4\% at 120 GeV and 6.2\% at 150 GeV.

Theoretical uncertainties on the background cross-sections do not affect the
results shown in the next section, because the background normalization and shape
are obtained through a fit to data.

\begin{table}[!htbp] 
  \renewcommand{\arraystretch}{1.3}
  \begin{center}
    \caption{Theoretical systematic uncertainties for the SM Higgs
      boson production cross section and branching fraction of the
      $H\to Z\gamma$ decay at $\sqrt{s} = 7$ and 8 TeV,
      for a Higgs boson mass of 125 GeV.}
    \label{tab:theory_uncertainties}
%\vspace{1mm}
    \begin{tabular}{l|cc|cc|cc|cc|cc|c}
      \hline\hline
$\sqrt{s}$ & \multicolumn{11}{c}{Systematic uncertainty (\%)}\\
%\hline
           & \multicolumn{2}{c}{$\sigma(gg\to H)$} & \multicolumn{2}{|c}{$\sigma$(VBF)} & \multicolumn{2}{|c}{$\sigma(WH)$} & \multicolumn{2}{|c}{$\sigma(ZH)$} & \multicolumn{2}{|c|}{$\sigma(t\bar tH)$} & $B(H\to Z\gamma)$ \\
         & scale & PDF & scale & PDF & scale & PDF & scale & PDF & scale & PDF & \\
\hline
7 TeV & $^{+7.1}_{-7.8}$  & $^{+7.6}_{-7.1}$  & {\small $\pm 0.3$} & $^{+2.5}_{-2.1}$ & $^{+0.2}_{-0.8}$ & {\small $\pm 3.5$} & $^{+1.4}_{-1.6}$ & {\small $\pm 3.5$} & $^{+3.3}_{-9.3}$  & {\small $\pm 8.5$} & $^{+9.0}_{-8.8}$ \\
8 TeV &$^{+7.3}_{-7.9}$  &$^{+7.5}_{-6.9}$   & {\small $\pm 0.2$} & $^{+2.6}_{-2.8}$ & $^{+0.1}_{-0.6}$ & {\small $\pm 3.4$} & $^{+1.5}_{-1.4}$ & {\small $\pm 3.5$} & $^{+3.9}_{-9.3}$  & {\small $\pm 7.8$} & $^{+9.0}_{-8.8}$ \\
  \hline\hline
    \end{tabular}
  \end{center}
\end{table}

\subsection{Experimental uncertainties}
The following sources of experimental systematic uncertainties on the expected
signal yields have been considered:
\begin{itemize}
\item \textbf{Luminosity:}
The uncertainty on the integrated luminosity for the 2011 data is 1.8\% 
and $\pm 3.6\%$ in 2012 \cite{ATLAS:2012roa}.
%
\item \textbf{Acceptance of the kinematic requirements:}
The total acceptance of \HToZg events is defined as the ratio of the total number
of \HToZg events that pass all selections divided by the initial number of \HToZg
events. The uncertainties related to the acceptance of the selection criteria are
estimated using Monte Carlo simulations. This yields an uncertainty of 4\%.
%
\item \textbf{Photon identification efficiency:} 
Recall that to have reliable comparison between data and theory one uses Monte Carlo
simulation to generate events by computer randomly following a distribution predicted
by theory. However, there are areas where simulation can not reproduce data exactly
because there is no theory reproducing reality perfectly. In these areas one
need to bring Monte Carlo simulations to the level of data by hand. This is done
with the help of scale factors (SF) which are extracted from variables that are
easy to analyze. One such quantity is the photon identification efficiency scale
factor, which quantifies the discrepancy between data and MC when it comes to
identifying photons in the ATLAS detector. At $\rts = 7 \TeV$, the signal
yield is recomputing by varying the photon identification efficiency scale factors
within their uncertainties and the relative variation is considered as a systematic
uncertainty. At $\rts = 8 \TeV$ a conservative estimate of the uncertainty on the
photon identification efficiency obtained from a comparison between data-driven
measurements and the simulated efficiencies is used. This amounts to 2.5\% for
$\et < 40 \GeV$ and for unconverted photons with $\abseta > 1.81$ and to 1.5\%
otherwise. The resulting uncertainty on the \HToZg selection efficiency is below
3\%.
%
\item \textbf{Photon and electron calorimeter isolation requirements:}
The signal efficiency defined as $N_{\text{selected}}/N_{\text{truth}}$ 
is measured from simulated events. 
To estimate the systematic uncertainty due to the uncertainty
on the efficiency of the isolation criteria, the signal efficiency is recomputed
by shifting, in the simulation, the photon and electron calorimeter isolation
energies by the average difference observed between data and Monte Carlo for
photons and electrons, selected either in di-photon enriched events or
in a control sample of electrons from $Z \to ee$. These differences are of the
order of 100 MeV for the topological-cluster based isolation. The systematic
uncertainty on the signal efficiency range between 0.2\% and 0.4\%.
%
\item \textbf{Photon and electron energy scales:}
The energy of electromagnetic particles is measured by the electromagnetic 
calorimeter essentially through a measurement of the amount light produced by
the particle's interactions with the calorimeter's material. However, there
are a number of effects that spoil the accuracy of the conversion of light into
a particle's energy. The particles may punch through the calorimeter without leaving
their energy inside, or they may hit parts of the device which are uninstrumented or
malfunctioning. Therefore, a correction to the energy measured in the 
electromagnetic calorimeter is applied known as an energy scale factor, which
rescales the measured energy to match the one the original particle had.

The uncertainty from the electromagnetic (photon and electron) energy
scales is assessed by varying the electromagnetic scale corrections (applied
to the data) within their uncertainties. The effects of the uncertainty fomr
the $Z \to ee$ calibration sample used to to extract the scale factors, of the
limited knowledge of the material, of the uncertainty on the pre-sampler energy 
scale and the low-$\pt$ scale factor uncertainties are evaluated. The total
uncertainty on the signal efficiency is around 0.2\% for events in which
the $Z$ boson candidate decays to muons and between 0.4\% and 1.2\% for
events in which the $Z$ boson candidate decays to electrons.
%
\item \textbf{Photon and electron energy resolution:}
The measurement of the electromagnetic energy only has a finite resolution.
The uncertainty from this electromagnetic energy resolution is estimated by
varying the resolution smearing corrections within its uncertainties and
observing the relative variation in the predicted signal yield. The estimated
uncertainty is smaller than 0.2\%.
%
\item \textbf{Electron trigger, reconstruction, and identification efficiency:}
The electron trigger, reconstruction, and identification efficiency uncertainties
are estimated by varying the efficiency scale factors applied to the simulation within
their uncertainties. The total uncertainty, for events in which the $Z$ boson
candidate decays to electrons, is around 3\%.
%
\item \textbf{Muon momentum scale and resolution:} 
The uncertainty of the efficiency of the $\pt > 10 \GeV$ cut (15 GeV for muons
tagged in the calorimeters) is estimated by varying the muon momentum
corrections in MC by their uncertainties. The effect is around 0.1\%.
%
\item \textbf{Muon trigger, reconstruction, and identification efficiency:}
The trigger, reconstruction, and identification muon efficiency uncertainties
are estimated by varying the efficiency scale factors within their uncertainties.
The total uncertainty, for events which $Z$ boson candidate decays to muons, is
below 1\%.
\end{itemize}
Other sources of uncertainties have been estimated by comparing the efficiencies
in data and Monte Carlo for control samples of leptons from $Z$ decays and found to
be negligible. The total relative uncertainty in the signal efficiency is around 5\%.

The following sources of experimental systematic uncertainties on the signal \dm
distribution have been considered:
\begin{itemize}
\item \textbf{Photon and energy scales:}
The signal \dm distribution is recomputed after varying the electromagnetic
energy scale corrections within their uncertainties, and the shift of the
peak position (0.2 GeV) is considered the systematic uncertainty.
%
\item \textbf{Photon and electron energy resolution:}
The signal \dm distribution is recomputed after varying the electromagnetic
smearing corrections within their uncertainties, and the relative variation of
its width is taken as a systematic uncertainty. It amounts to 2-4\% for events
in which the $Z$ boson candidate decays to muons and to 5\% for events in which the 
$Z$ boson candidate decays to electrons.
%
\item \textbf{Muon momentum scale:}
The signal \dm distribution is recomputed after varying the muon momentum scale
within its uncertainties, and the shift of the peak position is considered
as a systematic uncertainty. This uncertainty is found to be negligible.
%
\item \textbf{Muon momentum resolution:}
The signal \dm distribution is recomputed after varying the muon momentum smearing
corrections within their uncertainties, and the relative variation of its
width (0-1.5\%) is taken as a systematic uncertainty.
\end{itemize}

The list of the main sources of systematic uncertainties and their contributions
to the \HToZg expected signal yields and parameters of the signal \dm distributions
are listed in Table~\ref{tab:zg_syst_125_8_7tev} for $m_H = 125 \GeV$ and
$\rts = 8$ (7) TeV. The systematic uncertainties are profiled in the final
maximum likelihood fit to the data, as described in \refS{subsec:nuisance}. All
systematic uncertainties, except that on the luminosity, are treated as
correlated between $\rts = 7 \TeV$ and the $\rts = 8 \TeV$ analyses.

\begin{table}[!htbp]
\centering
\caption{Summary of the systematic uncertainties on the signal yield and
  invariant mass distribution for $m_H = 125$ GeV, at $\sqrt{s}=8 (7)$ TeV.}
  \label{tab:zg_syst_125_8_7tev} 
\small
 \begin{tabular}{cccc}
       \hline
       \hline
       \textbf{Systematic Uncertainty }           & $H \rightarrow Z(ee) \gamma$(\%)   & $H \rightarrow Z(\mu\mu) \gamma$(\%) \\
       \hline
       \textbf{Signal Yield}                      &                                    &                                      \\ 
       \hline
       \multicolumn{1}{l}{Luminosity}                                 & 3.6 (1.8)                        & 3.6 (1.8)          \\
       \multicolumn{1}{l}{Trigger efficiency}                         & 0.4 (0.2)                        & 0.8 (0.7)          \\
       \multicolumn{1}{l}{Acceptance of kinematic selection}          & 4.0 (4.0)                        & 4.0 (4.0)          \\
       \multicolumn{1}{l}{$\gamma$ identification efficiency}         & 2.9 (2.9)                        & 2.9 (2.9)          \\  
       \multicolumn{1}{l}{electron reconstruction and identification efficiency} & 2.7 (3.0)             &                    \\  
       \multicolumn{1}{l}{$\mu$ reconstruction and identification efficiency}    &                       & 0.6 (0.7)          \\  
       \multicolumn{1}{l}{$e/\gamma$ energy scale}                    & 1.4 (0.3)                        & 0.3 (0.2)          \\
%%       \multicolumn{1}{r}{Method uncertainty ($Z\to ee$)} & \multicolumn{1}{r}{1.4 (0.2)}   & \multicolumn{1}{r}{0.2 (0.1)} \\
%%       \multicolumn{1}{r}{Material uncertainty}   & \multicolumn{1}{r}{0.1 (0.1)}    & \multicolumn{1}{r}{0.2 (0.1)}        \\
%%       \multicolumn{1}{r}{Presampler energy scale}& \multicolumn{1}{r}{0.0 (0.1)}    & \multicolumn{1}{r}{0.0 (0.0)}        \\
%%       \multicolumn{1}{r}{Low pt}                 & \multicolumn{1}{r}{0.2 (0.2)}    & \multicolumn{1}{r}{0.2 (0.1)}        \\
       \multicolumn{1}{l}{$e/\gamma$ isolation}                       & 0.4 (0.3)                        & 0.4 (0.2)          \\  
       \multicolumn{1}{l}{$e/\gamma$ energy resolution}               & 0.2 (0.2)                        & 0.0 (0.0)          \\
       \multicolumn{1}{l}{$\mu$ momentum scale}                       &                                  & 0.1 (0.1)          \\  
       \multicolumn{1}{l}{$\mu$ momentum resolution}                  &                                  & 0.0 (0.1)          \\
       \hline
       \textbf{Signal $\Delta m$ resolution}      &                                    &                                      \\  
       \hline 
       \multicolumn{1}{l}{$e/\gamma$ energy resolution}               & 5.0 (5.0)                        & 2.4 (2.4)          \\ 
       \multicolumn{1}{l}{$\mu$ momentum resolution}                  &                                  & 0.0 (1.5)          \\
       \hline
       \textbf{Signal $\Delta m$ peak position} \\
       \hline
       \multicolumn{1}{l}{$e/\gamma$ energy scale}                    & 0.2 (0.2) GeV                    & 0.2 (0.2) GeV    \\
       \multicolumn{1}{l}{$\mu$ momentum scale}                       &                                  & negligible         \\
       \hline
       \hline
 \end{tabular}
\end{table}
