\section{Event Selection}
\label{sec:event}

Although the ATLAS detector measured billions of proton-proton collisions in 2011
and 2012 only a small fraction contain \HToZg decays. Therefore, an efficient
means of identifying these events using the tracking, particle identification, 
energy, and momentum information provided by the ATLAS detector is developed.
This section begins with a summary of the data samples used in this analysis
as well as any pre-analysis kinematic requirements applied to the data.
Next the algorithms used to reconstruct and identify the final state muons,
electrons, and photons from the crude information provided by the ATLAS detector
are outlined. Finally, the criteria used to interpolate back to the $Z$ and 
Higgs bosons is summarized. 

\subsection{Data Samples}
To ensure that the analyzed data is of high quality, events in which the ATLAS
detector is not fully operational, or shows data quality problems are excluded.
The resulting integrated luminosity after the trigger and data quality 
requirements corresponds to 20.7 \ifb (4.6 \ifb), with a relative uncertainty
of 3.6\% (1.8\%) at $\rts =$ 8 TeV ($\rts =$ 7 TeV) 
\cite{Aad:2011dr, Aad:2013ucp}.
Unless explicitly specified, the selection criteria
at $\rts = 8 \TeV$ and $\rts = 7 \TeV$ are identical. Values quoted in parentheses
correspond to the $\rts = 7 \TeV$ run period.


The data quality is determined by requiring the event to belong to a good runs list 
(GRL) generated by the ATLAS HSG2 group, and rejecting events with measurement 
errors in the lAr calorimeter, measurement errors in tile calorimeter, or are 
determined to be corrupt. 
The fraction of events removed in data by the GRL requirement
is around 7\% in 2011 data and 4-5\% in 2012 data. In addition, the fraction of
events with lAr, tile, or corrupt events is 0.3\% in 2011 and 0.5-0.6\% in 2012 
\cite{Abreu:1500529}.

In order to remain as unbiased as possible, the data is selected using the lowest
threshold, unprescaled single-lepton and di-lepton triggers. For the
single-muon trigger the transverse momentum, \pt, threshold is 24 (18) GeV, while
for the single-electron trigger the transverse energy, \et, threshold is 24 (20) GeV.
For the di-muon triggers, the thresholds are \pt $>$ 13 (10) GeV for each muon, while
for the di-electron triggers the thresholds are \et $>$ 12 GeV for each electron.
An asymmetric di-muon trigger is used at $\rts = 8 \TeV$ that requires one muon with 
a \pt $>$ 18 GeV and another with \pt $>$ 8 GeV. 
The trigger efficiency for signal
events passing the selection described is around 99\% for events in which the
$Z$ boson decays into $ee$ (the $ee\gamma$ channel) and 92\% for events where
the $Z$ boson decays to $\mu\mu$ pairs (the $\mu\mu\gamma$ channel). The lower
efficiency of the muon trigger is due to the reduced geometric acceptance of the muon
trigger system in the $|\eta| < 1.05$ region \cite{ATLAS-CONF-2013-009}.

In addition to an event trigger, the datasets are pre-filtered to 
reduce their size, allowing for a reasonable computing time.
The skims used were originally developed for data-driven photon identification
efficiency measurements using radiative $Z \to \ell\ell\gamma$ decays and
are based on loose kinematic and quality selections on the leptons. The
skims require at least one primary vertex with three associated tracks,
a lepton transverse momenta and photon transverse energy greater than 10 GeV,
{\tt loose++} electron identificaiton or {\tt loose} muon identification for
the lepton candidates, and at least one same flavor lepton pair with invariant mass
greater than 10 GeV. These filters retain roughly 3 - 4\% of the original events
\footnote{See this wiki for more details 
{\tt https://twiki.cern.ch/twiki/bin/viewauth/AtlasProtected/RadiativeZDAODs}.}.

\subsection{Monte Carlo Samples}
\label{subsec:mc}
In order to study the characteristics of both signal and background events 
in the data, simulated data known as Monte Carlo samples\footnote{This name
comes from the fact that the Monte Carlo integration method is used in generating
these simulated samples.} are produced. These simulated samples are produced
in the following three stages
\[
    \text{Parton level MC} \rightarrow \text{Shower MC} 
    \rightarrow \text{Experimental Simulation}.
\]
Parton level Monte Carlo are responsible for producing the hard scattering process
of interest, e.g. the parton-parton collision that produces a SM 
Higgs boson decaying to $Z\gamma$, followed by a decay of the $Z$ to 
a charged lepton pair. The shower Monte Carlo takes
any remaining quarks/gluons in the parton level MC and goes 
through a process of showering (QCD radiation in the form of gluons) and
hadronization to produce particle jets. Finally, the interaction of the resulting
particles with the detector materials is modeled with GEANT4 \cite{geant} and
the detector response is simulated. More details on the ATLAS event generation and
simulation infrastructure are provided in \cite{ATLASsimulation}.

The main backgrounds in this search originate from $Z+\gamma$ events either
from di-boson production with initial-state radiation (ISR)
in which a $q\bar q$ pair annihilate to form a $Z$ boson and one of the 
quarks radiate a photon ($q\bar q \to Z\gamma$), 
from final-state-radiate (FSR) in radiative $Z$ boson decays 
($Z \to \ell\ell\gamma$). or from parton-to-photon fragmentation, and production
of a $Z$ boson in association with a jet, and misidentification of a particle
jet as a photon. There is also a smaller contribution from  $t\bar t$ and $WZ$
events.

The various signal and background Monte Carlo samples used in the 2012 (2011) 
analysis are presented in \refT{tab:MCsamples_8TeV} (\ref{tab:MCsamples_7TeV}).

% created with maketableMClumifornote.sh
\begin{sidewaystable}[!htbp]
  \begin{center}
    \caption{Simulated samples at $\sqrt{s}=7$ TeV. The number of events listed in the
      second column is either the number of events in the D3PD (for unskimmed samples)
      or of events in the AOD before skimming (for skimmed samples).
      The cross sections for the signals are from Ref.~\cite{LHCHiggsCrossSectionWorkingGroup:2011ti}, 
    while for background samples the cross sections are those returned by the MC generators.
      The equivalent luminosity is computed as $N_{\rm evts}/(\sigma_{\rm gen}\varepsilon_{\rm filt})$.}
    \label{tab:MCsamples_7TeV}
    \tiny
       \begin{tabular}{ccccc}
         \hline
         \hline
         D3PD name &           Events  &    $\sigma_{\rm gen}$ & $\varepsilon_{\rm filt}$ & eq. lumi \\
                   &                   &    [nb]              &                        &  [nb$^{-1}$]   \\
         \hline
         \multicolumn{5}{c}{$H\to Z\gamma$ signal} \\
         \hline
%% Numbers with xsection from Yellow book
%% OLD MC
%% mc11\_7TeV.167520.PowHegPythia\_ggH120\_Zllgam.merge.NTUP\_PHOTON.e1465\_s1372\_s1370\_r3108\_r3109\_p1333  & 100000 & 1.866e-06 & 1 & 5.359e+10 \\
%% mc11\_7TeV.167521.PowHegPythia\_ggH125\_Zllgam.merge.NTUP\_PHOTON.e1465\_s1372\_s1370\_r3108\_r3109\_p1333  & 100000 & 2.382e-06 & 1 & 4.198e+10 \\
%% mc11\_7TeV.167522.PowHegPythia\_ggH130\_Zllgam.merge.NTUP\_PHOTON.e1465\_s1372\_s1370\_r3108\_r3109\_p1333  &  99999 & 2.788e-06 & 1 & 3.587e+10 \\
%% mc11\_7TeV.167530.PowHegPythia\_VBFH120\_Zllgam.merge.NTUP\_PHOTON.e1465\_s1372\_s1370\_r3108\_r3109\_p1333 &  49999 & 1.435e-07 & 1 & 3.484e+11 \\
%% mc11\_7TeV.167531.PowHegPythia\_VBFH125\_Zllgam.merge.NTUP\_PHOTON.e1465\_s1372\_s1370\_r3108\_r3109\_p1333 &  50000 & 1.897e-07 & 1 & 2.636e+11 \\
%% mc11\_7TeV.167532.PowHegPythia\_VBFH130\_Zllgam.merge.NTUP\_PHOTON.e1465\_s1372\_s1370\_r3108\_r3109\_p1333 &  50000 & 2.304e-07 & 1 & 2.170e+11 \\
%% NEW MC
%% mc11\_7TeV.167483.PowhegPythia8\_AU2CT10\_ggH115\_Zllgam.merge.NTUP\_PHOTON.e1749\_s1570\_s1571\_r3108\_r3109\_p1373 & 100000 & & 1 & \\
mc11\_7TeV.167480.PowhegPythia8\_AU2CT10\_ggH120\_Zllgam.merge.NTUP\_PHOTON.e1749\_s1570\_s1571\_r3108\_r3109\_p1373 & 100000 & 1.866e-06 & 1 & 5.359e+10 \\
mc11\_7TeV.167481.PowhegPythia8\_AU2CT10\_ggH125\_Zllgam.merge.NTUP\_PHOTON.e1749\_s1570\_s1571\_r3108\_r3109\_p1373 & 100000 & 2.382e-06 & 1 & 4.198e+10 \\
mc11\_7TeV.167482.PowhegPythia8\_AU2CT10\_ggH130\_Zllgam.merge.NTUP\_PHOTON.e1749\_s1570\_s1571\_r3108\_r3109\_p1373 &  99999 & 2.788e-06 & 1 & 3.587e+10 \\
mc11\_7TeV.167484.PowhegPythia8\_AU2CT10\_ggH135\_Zllgam.merge.NTUP\_PHOTON.e1749\_s1570\_s1571\_r3108\_r3109\_p1373 &  99999 & 3.005e-06 & 1 & 3.328e+10 \\
mc11\_7TeV.167485.PowhegPythia8\_AU2CT10\_ggH140\_Zllgam.merge.NTUP\_PHOTON.e1749\_s1570\_s1571\_r3108\_r3109\_p1373 &  99999 & 2.374e-06 & 1 & 4.212e+10 \\
mc11\_7TeV.167486.PowhegPythia8\_AU2CT10\_ggH145\_Zllgam.merge.NTUP\_PHOTON.e1749\_s1570\_s1571\_r3108\_r3109\_p1373 &  70000 & 1.922e-06 & 1 & 3.642e+10 \\
mc11\_7TeV.167487.PowhegPythia8\_AU2CT10\_ggH150\_Zllgam.merge.NTUP\_PHOTON.e1749\_s1570\_s1571\_r3108\_r3109\_p1373 &  99999 & 1.464e-06 & 1 & 6.831e+10 \\
%% mc11\_7TeV.167493.PowhegPythia8\_AU2CT10\_VBFH115\_Zllgam.merge.NTUP\_PHOTON.e1749\_s1570\_s1571\_r3108\_r3109\_p1373 & 100000 & & 1 & \\
mc11\_7TeV.167490.PowhegPythia8\_AU2CT10\_VBFH120\_Zllgam.merge.NTUP\_PHOTON.e1749\_s1570\_s1571\_r3108\_r3109\_p1373 & 100000 & 1.433e-07 & 1 & 6.978e+11 \\ 
mc11\_7TeV.167491.PowhegPythia8\_AU2CT10\_VBFH125\_Zllgam.merge.NTUP\_PHOTON.e1749\_s1570\_s1571\_r3108\_r3109\_p1373 & 100000 & 1.900e-07 & 1 & 5.263e+11 \\ 
mc11\_7TeV.167492.PowhegPythia8\_AU2CT10\_VBFH130\_Zllgam.merge.NTUP\_PHOTON.e1749\_s1570\_s1571\_r3108\_r3109\_p1373 & 100000 & 2.300e-07 & 1 & 4.348e+11 \\ 
mc11\_7TeV.167494.PowhegPythia8\_AU2CT10\_VBFH135\_Zllgam.merge.NTUP\_PHOTON.e1749\_s1570\_s1571\_r3108\_r3109\_p1373 & 100000 & 2.560e-07 & 1 & 3.906e+11\\ 
mc11\_7TeV.167495.PowhegPythia8\_AU2CT10\_VBFH140\_Zllgam.merge.NTUP\_PHOTON.e1749\_s1570\_s1571\_r3108\_r3109\_p1373 &  99899 & 2.083e-07 & 1 & 4.796e+11\\ 
mc11\_7TeV.167496.PowhegPythia8\_AU2CT10\_VBFH145\_Zllgam.merge.NTUP\_PHOTON.e1749\_s1570\_s1571\_r3108\_r3109\_p1373 & 100000 & 1.735e-07 & 1 & 5.764e+11\\ 
mc11\_7TeV.167497.PowhegPythia8\_AU2CT10\_VBFH150\_Zllgam.merge.NTUP\_PHOTON.e1749\_s1570\_s1571\_r3108\_r3109\_p1373 & 100000 & 1.356e-07 & 1 & 7.375e+11\\ 
         \hline
         \multicolumn{5}{c}{$Z\to \ell\ell\gamma$ background} \\
         \hline
mc11\_7TeV.145161.Sherpa\_Zeegamma\_3jets.merge.NTUP\_PHOTON.e1443\_s1372\_s1370\_r3108\_r3109\_p1373 & 399799 & 0.027344 & 1 & 1.46211e+07 \\
mc11\_7TeV.145162.Sherpa\_Zmumugamma\_3jets.merge.NTUP\_PHOTON.e1443\_s1372\_s1370\_r3108\_r3109\_p1373 & 399800 & 0.027301 & 1 & 1.46442e+07 \\
         \hline
         \multicolumn{5}{c}{$Z(\to \ell\ell) + jets$ background} \\
         \hline
mc11\_7TeV.147771.Sherpa\_CT10\_Zmumu.merge.NTUP\_PHOTON.e1443\_s1372\_s1370\_r3108\_r3109\_p1373      & 9995488 & 1.0452     & 1 & 9.56323e+06 \\
mc11\_7TeV.107650.AlpgenJimmyZeeNp0\_pt20.merge.NTUP\_PHOTON.e835\_s1299\_s1300\_r3043\_r2993\_p1373   & 6618284 & 0.6696     & 1 & 9.88394e+06 \\
mc11\_7TeV.107651.AlpgenJimmyZeeNp1\_pt20.merge.NTUP\_PHOTON.e835\_s1299\_s1300\_r3043\_r2993\_p1373   & 1334897 & 0.13452    & 1 & 9.92341e+06 \\
mc11\_7TeV.107652.AlpgenJimmyZeeNp2\_pt20.merge.NTUP\_PHOTON.e835\_s1299\_s1300\_r3043\_r2993\_p1373   & 2004195 & 0.040706   & 1 & 4.92359e+07 \\
mc11\_7TeV.107653.AlpgenJimmyZeeNp3\_pt20.merge.NTUP\_PHOTON.e835\_s1299\_s1300\_r3043\_r2993\_p1373   & 549949  & 0.011262   & 1 & 4.88323e+07 \\
mc11\_7TeV.107654.AlpgenJimmyZeeNp4\_pt20.merge.NTUP\_PHOTON.e835\_s1299\_s1300\_r3043\_r2993\_p1373   & 149948  & 0.0028447  & 1 & 5.27114e+07 \\
mc11\_7TeV.107655.AlpgenJimmyZeeNp5\_pt20.merge.NTUP\_PHOTON.e835\_s1299\_s1300\_r3043\_r2993\_p1373   & 50000   & 0.00075691 & 1 & 6.60581e+07 \\
mc11\_7TeV.147770.Sherpa\_CT10\_Zee.merge.NTUP\_PHOTON.e1443\_s1372\_s1370\_r3108\_r3109\_p1373        & 9998665 & 1.0458     & 1 & 9.56078e+06 \\
mc11\_7TeV.107660.AlpgenJimmyZmumuNp0\_pt20.merge.NTUP\_PHOTON.e835\_s1299\_s1300\_r3043\_r2993\_p1373 & 6615230 & 0.66956    & 1 & 9.87997e+06 \\
mc11\_7TeV.107661.AlpgenJimmyZmumuNp1\_pt20.merge.NTUP\_PHOTON.e835\_s1299\_s1300\_r3043\_r2993\_p1373 & 1334296 & 0.13455    & 1 & 9.91673e+06 \\
mc11\_7TeV.107662.AlpgenJimmyZmumuNp2\_pt20.merge.NTUP\_PHOTON.e835\_s1299\_s1300\_r3043\_r2993\_p1373 & 1999941 & 0.040642   & 1 & 4.92087e+07 \\
mc11\_7TeV.107663.AlpgenJimmyZmumuNp3\_pt20.merge.NTUP\_PHOTON.e835\_s1299\_s1300\_r3043\_r2993\_p1373 & 549896  & 0.011279   & 1 & 4.8754e+07 \\
mc11\_7TeV.107664.AlpgenJimmyZmumuNp4\_pt20.merge.NTUP\_PHOTON.e835\_s1299\_s1300\_r3043\_r2993\_p1373 & 150000  & 0.002837   & 1 & 5.28728e+07 \\
mc11\_7TeV.107665.AlpgenJimmyZmumuNp5\_pt20.merge.NTUP\_PHOTON.e835\_s1299\_s1300\_r3043\_r2993\_p1373 & 50000   & 0.00076123 & 1 & 6.56832e+07 \\
         \hline
         \multicolumn{5}{c}{$t\bar{t}$ background} \\
         \hline
mc11\_7TeV.109345.T1\_McAtNlo\_Jimmy\_2LeptonsMll60GeV.merge.NTUP\_PHOTON.e961\_s1310\_s1300\_r3043\_r2993\_p1373 & 499948 & 0.14519 & 0.081719 & 4.21371e+07 \\
mc11\_7TeV.109345.T1\_McAtNlo\_Jimmy\_2LeptonsMll60GeV.merge.NTUP\_PHOTON.e961\_s1310\_s1300\_r3043\_r2993\_p1373 & 499948 & 0.14519 & 0.081719 & 4.21371e+07 \\
         \hline
         \multicolumn{5}{c}{$WZ$ background} \\
         \hline
mc11\_7TeV.128811.SherpaWZlllnu.merge.NTUP\_PHOTON.e950\_s1310\_s1300\_r3043\_r2993\_p1373 & 299950 & 0.00036164 & 1 & 8.29416e+08 \\
mc11\_7TeV.128811.SherpaWZlllnu.merge.NTUP\_PHOTON.e950\_s1310\_s1300\_r3043\_r2993\_p1373 & 299950 & 0.00036164 & 1 & 8.29416e+08 \\
         \hline\hline
       \end{tabular}
  \end{center}
\end{sidewaystable}

% created with maketableMClumifornote.sh
\begin{sidewaystable}[!htbp]
  \begin{center}
    \caption{Simulated samples at $\sqrt{s}=8$ TeV. The number of events listed in the
      second column is either the number of events in the D3PD (for unskimmed samples)
      or of events in the AOD before skimming (for skimmed samples).
      The cross sections for the signals are from Ref.~\cite{LHCHiggsCrossSectionWorkingGroup:2012vm},
    while for background samples the cross sections are those returned by the MC generators.
      The equivalent luminosity is computed as $N_{\rm evts}/(\sigma_{\rm gen}\varepsilon_{\rm filt})$.}
    \label{tab:MCsamples_8TeV}
    \tiny
       \begin{tabular}{ccccc}
         \hline
         \hline
         D3PD name &           Events  &    $\sigma_{\rm gen}$ & $\varepsilon_{\rm filt}$ & eq. lumi \\
                   &                   &    [nb]              &                        &  [nb$^{-1}$]   \\
         \hline
         \multicolumn{5}{c}{$H\to Z\gamma$ signal} \\
         \hline
%% Numbers with xsection from Yellow book
%% OLD MC
%%         mc12\_8TeV.167480.PowhegPythia8\_AU2CT10\_ggH120\_Zllgam.merge.NTUP\_PHOTON.e1464\_s1499\_s1504\_r3658\_r3549\_p1211 & 100000 & 2.368e-06 & 1 & 4.223e+10 \\
%%         mc12\_8TeV.167481.PowhegPythia8\_AU2CT10\_ggH125\_Zllgam.merge.NTUP\_PHOTON.e1464\_s1499\_s1504\_r3658\_r3549\_p1211 & 100000 & 3.036e-06 & 1 & 3.294e+10 \\
%%         mc12\_8TeV.167482.PowhegPythia8\_AU2CT10\_ggH130\_Zllgam.merge.NTUP\_PHOTON.e1464\_s1499\_s1504\_r3658\_r3549\_p1211 & 100000 & 3.558e-06 & 1 & 2.811e+10 \\
%%         mc12\_8TeV.167490.PowhegPythia8\_AU2CT10\_VBFH120\_Zllgam.merge.NTUP\_PHOTON.e1464\_s1499\_s1504\_r3658\_r3549\_p1211 & 39999 & 1.849e-07 & 1 & 2.163e+11 \\
%%         mc12\_8TeV.167491.PowhegPythia8\_AU2CT10\_VBFH125\_Zllgam.merge.NTUP\_PHOTON.e1464\_s1499\_s1504\_r3658\_r3549\_p1211 & 49900 & 2.457e-07 & 1 & 2.031e+11 \\
%%         mc12\_8TeV.167492.PowhegPythia8\_AU2CT10\_VBFH130\_Zllgam.merge.NTUP\_PHOTON.e1464\_s1499\_s1504\_r3658\_r3549\_p1211 & 50000 & 2.973e-07 & 1 & 1.682e+11 \\

%% NEW MC
%%mc12\_8TeV.167483.PowhegPythia8\_AU2CT10\_ggH115\_Zllgam.merge.NTUP\_PHOTON.e1746\_s1581\_s1586\_r3658\_r3549\_p1344 &      0 & & 1 & \\
mc12\_8TeV.167480.PowhegPythia8\_AU2CT10\_ggH120\_Zllgam.merge.NTUP\_PHOTON.e1746\_s1581\_s1586\_r3658\_r3549\_p1344 & 100000 & 2.368e-06 & 1 & 4.223e+10 \\
mc12\_8TeV.167481.PowhegPythia8\_AU2CT10\_ggH125\_Zllgam.merge.NTUP\_PHOTON.e1746\_s1581\_s1586\_r3658\_r3549\_p1344 &  99999 & 3.035e-06 & 1 & 3.294e+10 \\
mc12\_8TeV.167482.PowhegPythia8\_AU2CT10\_ggH130\_Zllgam.merge.NTUP\_PHOTON.e1746\_s1581\_s1586\_r3658\_r3549\_p1344 &  99999 & 3.558e-06 & 1 & 2.811e+10 \\
mc12\_8TeV.167484.PowhegPythia8\_AU2CT10\_ggH135\_Zllgam.merge.NTUP\_PHOTON.e1746\_s1581\_s1586\_r3658\_r3549\_p1344 &  99999 & 3.848e-06 & 1 & 2.599e+10 \\
mc12\_8TeV.167485.PowhegPythia8\_AU2CT10\_ggH140\_Zllgam.merge.NTUP\_PHOTON.e1746\_s1581\_s1586\_r3658\_r3549\_p1344 &  99998 & 3.046e-06 & 1 & 3.283e+10 \\
mc12\_8TeV.167486.PowhegPythia8\_AU2CT10\_ggH145\_Zllgam.merge.NTUP\_PHOTON.e1746\_s1581\_s1586\_r3658\_r3549\_p1344 & 100000 & 2.475e-06 & 1 & 4.040e+10 \\
mc12\_8TeV.167487.PowhegPythia8\_AU2CT10\_ggH150\_Zllgam.merge.NTUP\_PHOTON.e1746\_s1581\_s1586\_r3658\_r3549\_p1344 &  89999 & 1.888e-06 & 1 & 4.767e+10 \\
%%mc12\_8TeV.167493.PowhegPythia8\_AU2CT10\_VBFH115\_Zllgam.merge.NTUP\_PHOTON.e1746\_s1581\_s1586\_r3658\_r3549\_p1344 &      0 & & 1 & \\
mc12\_8TeV.167490.PowhegPythia8\_AU2CT10\_VBFH120\_Zllgam.merge.NTUP\_PHOTON.e1746\_s1581\_s1586\_r3658\_r3549\_p1344 &  79999 & 1.848e-07 & 1 & 4.329e+11 \\
mc12\_8TeV.167491.PowhegPythia8\_AU2CT10\_VBFH125\_Zllgam.merge.NTUP\_PHOTON.e1746\_s1581\_s1586\_r3658\_r3549\_p1344 &  99900 & 2.454e-07 & 1 & 4.071e+11 \\
mc12\_8TeV.167492.PowhegPythia8\_AU2CT10\_VBFH130\_Zllgam.merge.NTUP\_PHOTON.e1746\_s1581\_s1586\_r3658\_r3549\_p1344 &  99999 & 2.975e-07 & 1 & 3.361e+11 \\
mc12\_8TeV.167494.PowhegPythia8\_AU2CT10\_VBFH135\_Zllgam.merge.NTUP\_PHOTON.e1746\_s1581\_s1586\_r3658\_r3549\_p1344 & 100000 & 3.319e-07 & 1 & 3.013e+11 \\
mc12\_8TeV.167495.PowhegPythia8\_AU2CT10\_VBFH140\_Zllgam.merge.NTUP\_PHOTON.e1746\_s1581\_s1586\_r3658\_r3549\_p1344 &  99997 & 2.707e-07 & 1 & 3.694e+11 \\
mc12\_8TeV.167496.PowhegPythia8\_AU2CT10\_VBFH145\_Zllgam.merge.NTUP\_PHOTON.e1746\_s1581\_s1586\_r3658\_r3549\_p1344 &  99500 & 2.261e-07 & 1 & 4.401e+11 \\
mc12\_8TeV.167497.PowhegPythia8\_AU2CT10\_VBFH150\_Zllgam.merge.NTUP\_PHOTON.e1746\_s1581\_s1586\_r3658\_r3549\_p1344 &  79999 & 1.771e-07 & 1 & 4.517e+11 \\
%%
         mc12\_8TeV.167674.MCFMPythia8\_AU2CTEQ6L1\_ggH120\_Zgamma\_ee.merge.NTUP\_PHOTON.e1619\_s1581\_s1586\_r3658\_r3549\_p1211   & 100000 & 7.893e-07 & 1 & 1.267e+11 \\
         mc12\_8TeV.167675.MCFMPythia8\_AU2CTEQ6L1\_ggH125\_Zgamma\_ee.merge.NTUP\_PHOTON.e1619\_s1581\_s1586\_r3658\_r3549\_p1211   & 100000 & 1.012e-06 & 1 & 9.881e+10 \\
         mc12\_8TeV.167676.MCFMPythia8\_AU2CTEQ6L1\_ggH130\_Zgamma\_ee.merge.NTUP\_PHOTON.e1619\_s1581\_s1586\_r3658\_r3549\_p1211   & 100000 & 1.186e-06 & 1 & 8.432e+10 \\
         mc12\_8TeV.167622.MCFMPythia8\_AU2CTEQ6L1\_ggH120\_Zgamma\_mumu.merge.NTUP\_PHOTON.e1619\_s1581\_s1586\_r3658\_r3549\_p1211 & 100000 & 7.893e-07 & 1 & 1.267e+11 \\
         mc12\_8TeV.167623.MCFMPythia8\_AU2CTEQ6L1\_ggH125\_Zgamma\_mumu.merge.NTUP\_PHOTON.e1619\_s1581\_s1586\_r3658\_r3549\_p1211 & 100000 & 1.012e-06 & 1 & 9.881e+10 \\
         mc12\_8TeV.167624.MCFMPythia8\_AU2CTEQ6L1\_ggH130\_Zgamma\_mumu.merge.NTUP\_PHOTON.e1619\_s1581\_s1586\_r3658\_r3549\_p1211 & 100000 & 1.186e-06 & 1 & 8.432e+10 \\
         \hline
         \multicolumn{5}{c}{$Z\to \ell\ell\gamma$ background} \\
         \hline
         mc12\_8TeV.145161.Sherpa\_CT10\_eegammaPt10.merge.NTUP\_PHOTON.e1434\_s1499\_s1504\_r3658\_r3549\_p1211 & 1184995 & 0.032261 & 1 & 3.67315e+07 \\
         mc12\_8TeV.145162.Sherpa\_CT10\_mumugammaPt10.merge.NTUP\_PHOTON.e1434\_s1499\_s1504\_r3658\_r3549\_p1211 & 1199997 & 0.032317 & 1 & 3.71321e+07 \\
% I think we do not use anywhere...       
%         mc12\_8TeV.129250.PowhegPythia8\_AU2CT10\_Zee\_2Lepton1Photon.merge.NTUP\_PHOTON.e1444\_s1499\_s1504\_r3658\_r3549\_p1208\_p1210 & 4999981 & 1.1099 & 0.039685 & 1.13516e+08 \\
%         mc12\_8TeV.129251.PowhegPythia8\_AU2CT10\_Zmumu\_2Lepton1Photon.merge.NTUP\_PHOTON.e1444\_s1499\_s1504\_r3658\_r3549\_p1208\_p1210 & 0 & 1.1099 & 0.02202 & 0 \\
         \hline
         \multicolumn{5}{c}{$Z(\to \ell\ell) + jets$ background} \\
         \hline
         mc12\_8TeV.147770.Sherpa\_CT10\_Zee.merge.NTUP\_PHOTON.e1161\_s1469\_s1470\_r3542\_r3549\_p1208\_p1210 & 9999568 & 1.2079 & 1 & 8.27847e+06 \\
         mc12\_8TeV.107650.AlpgenJimmy\_AUET2CTEQ6L1\_ZeeNp0.merge.NTUP\_PHOTON.e1218\_s1469\_s1470\_r3542\_r3549\_p1208\_p1210 & 6604283 & 0.71177 & 1 & 9.27868e+06 \\
         mc12\_8TeV.107651.AlpgenJimmy\_AUET2CTEQ6L1\_ZeeNp1.merge.NTUP\_PHOTON.e1218\_s1469\_s1470\_r3542\_r3549\_p1208\_p1210 & 1329994 & 0.15517 & 1 & 8.57121e+06 \\
         mc12\_8TeV.107652.AlpgenJimmy\_AUET2CTEQ6L1\_ZeeNp2.merge.NTUP\_PHOTON.e1218\_s1469\_s1470\_r3542\_r3549\_p1208\_p1210 & 404798 & 0.048745 & 1 & 8.3044e+06 \\
         mc12\_8TeV.107653.AlpgenJimmy\_AUET2CTEQ6L1\_ZeeNp3.merge.NTUP\_PHOTON.e1218\_s1469\_s1470\_r3542\_r3549\_p1208\_p1210 & 109998 & 0.014225 & 1 & 7.73272e+06 \\
         mc12\_8TeV.107654.AlpgenJimmy\_AUET2CTEQ6L1\_ZeeNp4.merge.NTUP\_PHOTON.e1218\_s1469\_s1470\_r3542\_r3549\_p1208\_p1210 & 30000 & 0.0037595 & 1 & 7.97978e+06 \\
         mc12\_8TeV.107655.AlpgenJimmy\_AUET2CTEQ6L1\_ZeeNp5.merge.NTUP\_PHOTON.e1218\_s1469\_s1470\_r3542\_r3549\_p1208\_p1210 & 10000 & 0.0010945 & 1 & 9.13659e+06 \\
         mc12\_8TeV.147771.Sherpa\_CT10\_Zmumu.merge.NTUP\_PHOTON.e1161\_s1469\_s1470\_r3542\_r3549\_p1208\_p1256 & 9999975 & 1.2078 & 1 & 8.2795e+06 \\
         mc12\_8TeV.107660.AlpgenJimmy\_AUET2CTEQ6L1\_ZmumuNp0.merge.NTUP\_PHOTON.e1218\_s1469\_s1470\_r3542\_r3549\_p1208\_p1256 & 6609982 & 0.71211 & 1 & 9.28225e+06 \\
         mc12\_8TeV.107661.AlpgenJimmy\_AUET2CTEQ6L1\_ZmumuNp1.merge.NTUP\_PHOTON.e1218\_s1469\_s1470\_r3542\_r3549\_p1208\_p1256 & 1334897 & 0.15477 & 1 & 8.62504e+06 \\
         mc12\_8TeV.107662.AlpgenJimmy\_AUET2CTEQ6L1\_ZmumuNp2.merge.NTUP\_PHOTON.e1218\_s1469\_s1470\_r3542\_r3549\_p1208\_p1256 & 404897 & 0.048912 & 1 & 8.27807e+06 \\
         mc12\_8TeV.107663.AlpgenJimmy\_AUET2CTEQ6L1\_ZmumuNp3.merge.NTUP\_PHOTON.e1218\_s1469\_s1470\_r3542\_r3549\_p1208\_p1256 & 110000 & 0.014226 & 1 & 7.73232e+06 \\
         mc12\_8TeV.107664.AlpgenJimmy\_AUET2CTEQ6L1\_ZmumuNp4.merge.NTUP\_PHOTON.e1218\_s1469\_s1470\_r3542\_r3549\_p1208\_p1256 & 29999 & 0.0037838 & 1 & 7.92827e+06 \\
         mc12\_8TeV.107665.AlpgenJimmy\_AUET2CTEQ6L1\_ZmumuNp5.merge.NTUP\_PHOTON.e1218\_s1469\_s1470\_r3542\_r3549\_p1208\_p1256 & 10000 & 0.0011148 & 1 & 8.97022e+06 \\
         \hline
         \multicolumn{5}{c}{$t\bar{t}$ background} \\
         \hline
         mc12\_8TeV.110001.McAtNloJimmy\_CT10\_ttbar\_dilepton.merge.NTUP\_PHOTON.e1193\_s1469\_s1470\_r3542\_r3549\_p1208\_p1210 & 7694555 & 0.02181 & 1 & 3.528e+08 \\
         mc12\_8TeV.110001.McAtNloJimmy\_CT10\_ttbar\_dilepton.merge.NTUP\_PHOTON.e1193\_s1469\_s1470\_r3542\_r3549\_p1208\_p1256 & 7694555 & 0.02181 & 1 & 3.528e+08 \\
         \hline
         \multicolumn{5}{c}{$WZ$ background} \\
         \hline
         mc12\_8TeV.129486.PowhegPythia8\_AU2CT10\_WZ\_W11Z11\_mll0p250d0\_2LeptonFilter5.merge.NTUP\_PHOTON.e1300\_s1469\_s1470\_r3542\_r3549\_p1208\_p1210 & 189899 & 0.0009795 & 0.29694 & 6.52904e+08 \\
         mc12\_8TeV.129487.PowhegPythia8\_AU2CT10\_WZ\_W11Z13\_mll0p4614d0\_2LeptonFilter5.merge.NTUP\_PHOTON.e1300\_s1469\_s1470\_r3542\_r3549\_p1208\_p1256 & 190000 & 0.000639 & 0.35302 & 8.42274e+08 \\
         \hline\hline
       \end{tabular}
  \end{center}
\end{sidewaystable}

\subsection{Primary Vertex Definition}
Events that pass the selection described below always have at least one
reconstructed primary vertex. These primary vertices are determined from fits
to tracks reconstructed in the inner detector and consistent with a common origin.
The interactions produced at the LHC are a result of collisions between the quarks
and gluons that make up the protons in the beams. These QCD processes are
classified as hard or soft depending on the amount of momentum transferred 
between the interacting particles. Of the roughly 20 proton-proton interactions
that occur during each bunch crossing, almost all of them are soft QCD processes.
This means there is very low momentum transfer in the collision, 
so mostly consist of glancing collisions between the quarks. 
However, the QCD process that produces a Higgs boson 
(gluon-gluon fusion) is a hard interaction with high momentum transfer.
In order to ensure that the primary vertex originated from the original hard 
interaction, the vertex is chosen with the largest sum of the 
squared transverse momenta of the tracks associated to it.

\subsection{Final State Particle Reconstruction and Selection}
Recall that two oppositely charged leptons ($\ell = e, \mu$) and a photon ($\gamma$)
make up the final state particles of a \HToZg event. This section
contains information regarding the reconstruction and identification of these 
particles from their energy deposition in the calorimeters and their tracking
information provided by the inner detector and the muon spectrometer.
In addition, the kinematic requirements enforced in order to select $\ell\ell\gamma$
triples originating from a Higgs decay are outlined. The sections proceed
as follows: muons, electrons, then photons.

\subsubsection*{Muon Reconstruction and Identification}
In ATLAS four kinds of muon candidates are distinguished depending on the
way they are reconstructed: stand-alone muons, combined muons, segment tagged muons,
and calorimeter tagged muons\cite{ATLAS-CONF-2010-064}.
\begin{itemize}
    \item \textbf{Stand-alone muons (SA):} SA muons are reconstructed entirely from
    tracking information provided by the muon spectrometer. The muons momentum
    is extrapolated back to the interaction point while taking into account the
    deviating effects from multiple coulomb scattering and energy loss in the 
    transversed material. SA muons utilize the extended coverage of the muon 
    spectrometer 
    and are used in the forward region ($2.5 < |\eta| < 2.7$) outside the 
    inner detector's coverage. Their use increases the overall acceptance 
    of the analysis. To increase the purity of these tracks, hits
    in all three muon spectrometer components where they are expected to pass
    are required.

    \item \textbf{Combined muons (CB):} The momentum information recorded by
    the MS is combined with the momentum measured in the inner detector. The
    muon trajectory in the inner detector also provides information about the 
    impact parameter of the muon trajectory with respect to the primary vertex.

    \item \textbf{Segment tagged muon (ST):} A trajectory in the inner detector
    is identified as a muon if the trajectory extrapolated to the muon spectrometer
    can be associated with straight track segments in the precision muon chambers,
    i.e the muon drift tubes or the cathode strip chambers.

    \item \textbf{Calorimeter tagged muon (CT):} A trajectory in the inner detector
    is identified as a muon if the associated energy deposition in the calorimeters
    is compatible with the hypothesis of a minimizing ionizing particle\footnote{
    A minimum ionizing particle (MIP) is defined as a particle whose rate 
    of mean energy loss through matter is close to the minimum. 
    The EM calorimeter is designed such that muons are MIPs, so this information
    can be used to identify muons in the calorimeter.}.
    CT muons are used to cover the region
    $|\eta| \approx 0$ which is not equipped with muon chambers. The purity
    of these muons is further increased by requiring their
    transverse momentum exceed 15 GeV.
\end{itemize}
In this analysis, CB, ST or SA muons identified with the STACO reconstruction 
algorithm as well as CT muons are used.

\subsubsection*{Muon Selection}
In the center of the barrel region ($|\eta| < 0.1$), which lacks muon spectrometer
coverage, calorimeter tagged muons with $\pt > 15 \GeV$ are used. All muon candidates
identified as muons using the information from the muon spectrometer (combined,
stand-alone, segment tagged) are required  to have a transverse momentum 
$\pt > 10 \GeV$. Each of these muons is used in the region where they have
the highest purity: combined and stand-alone muons are required to have $|\eta| < 2.7$
, and stand-alone muons are required to have $2.5 < |\eta| < 2.7$. The inner
detector tracks associated to muons that are identified inside the ID acceptance are
required to have a minimum number of associated hits in each of the ID sub-detectors
(to ensure good track reconstruction). 
%A summary of the inner-detector hit 
%requirements is summarized in \refT{tab:muonidhits}. 
In addition, the inner
detector tracks are required to have transverse (longitudinal) impact parameter
$d_0$ ($z_0$) with respect to the primary vertex, smaller than 1 mm (10 mm).

% \begin{table}[hbt]
%    \centering
%    \begin{tabular}{cc}
%       \hline
%       \hline
%       \multicolumn{2}{c}{2012} \\
%       \hline
%       ID Si hit requirement & expectBLayerHit=false or numberOfBLayerHits $\geq$ 1 \\
%                             & No. of pixel hits $+$ No. of crossed dead pixel sensors $ > $ 0 \\
%                             & No. of SCT hits $+$ No. of crossed dead SCT sensors $ > $ 4 \\
%                             & No. of pixel holes $+$ No. of SCT holes $ < $ 3 \\
%       TRT hit requirements: $0.1 < |\eta|\le1.9$& Hits + Outliers $>$ 5 \& $\frac{\rm{Outliers}}{\rm{Hits + outliers}}<0.9 $\\
% TRT hit requirements: $|\eta| < 0.1 $ or $|\eta|\geq1.9$ & if (Hits + Outliers $>$ 5):  $\frac{\rm{Outliers}}{\rm{Hits + outliers}} <0.9 $\\
% \hline
% \hline
%       \multicolumn{2}{c}{2011} \\
%       \hline
%       ID Si hit requirement & expectBLayerHit=false or numberOfBLayerHits $\geq$ 1 \\
%                             & No. of pixel hits $+$ No. of crossed dead pixel sensors $ > $ 1 \\
%                             & No. of SCT hits $+$ No. of crossed dead SCT sensors $ > $ 5 \\
%                             & No. of pixel holes $+$ No. of SCT holes $ < $ 3 \\
%       TRT hit requirements: $|\eta|<1.9$& Hits + Outliers $>$ 5 \& $\frac{\rm{Outliers}}{\rm{Hits + outliers}}<0.9 $\\
% TRT hit requirements: $|\eta|\geq1.9$ & if (Hits + Outliers $>$ 5):  $\frac{\rm{Outliers}}{\rm{Hits + outliers}} <0.9 $\\
% \hline
% \hline
%   \end{tabular}
%    \caption{List of Inner Detector hit requirements for the muon
%      tracks reconstructed in the ID.\label{tab:IDhits}}
%    \label{tab:muonidhits}
% \end{table}
 
\subsubsection*{Electron and Photon Reconstruction}
The electron and photon reconstruction are seeded from clusters of energy deposited
in the electromagnetic calorimeter. The reconstruction is designed to separate 
electrons, unconverted photons, and photons which have converted in the detector
material to electron-positron pairs. The clusters are matched to tracks and to 
conversion vertex candidates, which have 
been reconstructed in the inner detector and extrapolated to the second layer of
the calorimeter \cite{ATL-PHYS-PUB-2011-007}.
Electromagnetic clusters with a matching track in the inner-detector are classified
as electrons. The interaction of photons with the ATLAS detector can be
classified into two main categories: converted and unconverted photons. Converted
photons are classified by the presence of at least one ID track matching an
electromagnetic cluster originating from a vertex in the tracking volume. Clusters
without a matching track are classified as unconverted photons. Tracks matched
to electron candidates (and, for 8 TeV data, from photon conversions) and having
enough associated hits in the silicon detectors are fitted using a Gaussian-Sum
Filter, which accounts for bremsstrahlung energy loss \cite{ATLAS-CONF-2012-047}.
The energies of the clusters are calibrated, separately for electrons, converted
and unconverted photon candidates, to account for energy losses upstream of the 
calorimeter and for energy leakage outside of the cluster. In data, the calibration 
is refined by applying $\eta$-dependent correction factors, which are of the order
of $\pm 1\%$, determined from data-to-MC comparisons of the Z boson mass peak in
$\Zboson \to \epem$ events \cite{Aad:2011mk}. 
In addition, the energy measurement of converted
photons is improved with corrections, evaluated as a function of the conversion
radius (i.e. the distance from the beam-line at which the photon converts into an
\epem pair) based on dedicated Monte-Carlo simulation studies.

\subsubsection*{Electron Selection}
Electron candidates are required to have a transverse energy greater than 10 GeV,
pseudorapidity $|\eta| < 2.47$, and a well-reconstructed ID track pointing to an 
electromagnetic calorimeter cluster. The cluster should satisfy a set of 
identification criteria that require the longitudinal and transverse shower
profiles to be consistent with those expected for electromagnetic showers (
{\tt loose++} in ATLAS jargon) \cite{Aad:2011mk}. The electron four-momentum is formed
using the energy measured by the electromagnetic 
calorimeter and the track azimuth
and pseudorapidity measured in the inner detector. The electron track is required
to have a hit in the $b$-layer when passing through an active $b$-layer module
(this reduces fake electron candidates from photon conversions)
and is also required to have a longitudinal impact parameter, with respect to
the primary vertex, smaller than 10 mm.

\subsubsection*{Photon Selection}
Photon candidates are required to have a transverse energy greater than 15 GeV.
In order to exploit the fine segmentation of the first layer of the electromagnetic
calorimeter to discriminate between genuine prompt photons (photons directly produced
by parton-parton collisions) and fake photons with jets, the photon candidate 
pseudorapidity must satisfy $\abseta < 1.37$ or $1.52 < \abseta < 2.37$. The
photon four-momentum is formed using the energy measured by the calorimeter in the
direction determined by the nominal interaction point and the energy-weighted
barycenter of all cluster cells in the second layer of the calorimeter. Photons
reconstructed near regions of the calorimeter affected by read-out or high-voltage 
failures are not accepted. The identification of photons is performed through a
cut-based selection based on shower shapes measured in the first two longitudinal
layers of the electromagnetic calorimeter and on the leakage in the hadronic
calorimeter. To further suppress hadronic background, an isolation requirement is 
applied. The calorimeter isolation transverse energy $E_{T}^{\text{iso}}$ is
estimated by summing the transverse energy of three-dimensional positive-energy
topological clusters reconstructed in the electromagnetic and hadronic 
calorimeters in a cone of $\Delta R = 0.4$ around the photon candidate,
where the region within $0.125 \times 0.175$ in $\eta \times \phi$ around the
photon barycenter is excluded. The isolation energy is corrected by subtracting
the estimated contributions from leakage of the photon energy outside of the
excluded region and, using the technique described in 
Refs. \cite{Cacciari:2008gn, Cacciari:2009dp},
from the underlying event and additional $pp$ interactions. Photon candidates
are required to have an isolation transverse energy less than 4 GeV.

\subsubsection*{Overlap Removal}
An overlap removal between the electrons and muons that pass all selection
criteria and share the same inner detector track within a cone of $\Delta R < 0.02$
is performed: if the muon is identified by the MS, then the electron candidate is 
discarded, otherwise the muon candidate is rejected. Photon candidates that are
within $\Delta R < 0.3$ of a selected electron or muon candidate are also rejected,
thus suppressing background from final state radiation (FSR) $Z+\gamma$ events.

\subsection{$Z(\ll)$ Reconstruction and Selection}
$Z$ boson candidates are reconstructed from paris of same-flavor, opposite-sign
leptons passing the previous selections. Muon pairs are required to have no more
than one muon identified either without the muon spectrometer or in the region
outside the ID acceptance (one combined muon is required). In the case of
multiple $Z$ candidates in the same event, the candidate with the invariant
mass closest to the Z boson's mass ($m_Z = 91.2$ \cite{Beringer:1900zz}) is selected.
The leptons used in the reconstruction of the selected $Z$ boson candidate are also
required to match the trigger object(s) of the (single or double) lepton trigger
used to select the event. Finally, an invariant mass requirement $\mass\ell > 65 \GeV$
is applied in order to remove biases in the simulated $Z$+jet samples which have a 
generator-level di-lepton invariant mass cut with lower threshold.

\subsection{$H(Z\gamma)$ Reconstruction and Selection}
Higgs boson candidates are reconstructed from the combination of the $Z$ candidate
with invariant mass close to the $Z$ mass and the photons with the largest transverse
energy. Track and calorimeter isolation requirements are applied to the leptons
forming the Higgs boson candidate. The normalized track isolation is defined as the 
sum of the transverse momenta of tracks, $\sum \pt$ inside a cone of $\Delta R < 0.2$
around the lepton, excluding the lepton track, divided by the lepton \pt. Each lepton
is required to have a normalized track isolation smaller than 0.15. The normalized
calorimetric isolation for electrons, computed as the sum of the positive-energy
topological clusters with a reconstructed barycenter falling in a cone of
$\Delta R < 0.2$ around the candidate electron cluster divided by the electron \et,
must be lower than 0.2. The cells with $0.125 \times 0.175$ in $\eta \times \phi$
around the electron barycenter are excluded. The calorimetric isolation 
$E_{\text{T}}^{\text{cone}}$ for muons is computed from the sum of the transverse
energy of the calorimeter cells inside a cone of $\Delta R < 0.2$ around the muon
direction. The transverse energy of the cells inside a predefined region
around the points where the muon track crosses the different layers of the
calorimeter and accounts for the muon energy loss itself is subtracted. Muons
are required to have a normalized calorimetric isolation 
$E_{\text{T}}^{\text{cone}}/\pt$ less than 0.3 (0.15 in the case of muons without an
ID track). For both the track and calorimeter based isolation any contributions
arising from the other lepton from the candidate $Z$ decay are subtracted.
Additional track impact significance selections are also applied to the leptons
from the $Z$ candidate decay. The transverse impact parameter significance 
$|d_0|/\sigma_{d_0}$ of the ID track associated to leptons within the acceptance of
the inner detector is required to be less than 3.5 and 6.5 for muons and electrons,
respectively. The electron impact parameter is affected by bremsstrahlung and it
thus has a broader distribution. The isolation criteria, as well as the impact
significance selections applied in this analysis, have been optimized in the search
for the SM Higgs boson in the $H \to ZZ^* \to 4\ell$ channel 
\cite{ATLAS-CONF-2012-169}.

Finally, to suppress events from FSR $Z \to \ell\ell\gamma$, the di-lepton
invariant mass is required to be larger than the PDG value of the $Z$ boson mass
minus 10 GeV. This requirement also reduces the contribution to the signal
from internal photon conversions in $H \to \gamma\gamma$ to a negligible level 
\cite{Firan:1032388}.

The same reconstruction algorithms and selection criteria are used for
simulated events. The simulation is corrected to take into account known data-MC
differences on photon and lepton efficiencies and energy or momentum resolution.

After applying all selection criteria, the number of \HToZg candidates in the
$\rts = 8 \TeV$ ($\rts = 7 \TeV$) data sample is 13978 (1927) in the 
$Z \to ee\gamma$ channel and 16678 (2621) in the $Z \to \mu\mu\gamma$
channel. The larger number of events in the muon channel in data (and simulation)
is due to the larger acceptance of the muon spectrometer (up to $|\eta| = 2.7$)
and the higher selection efficiency for muons than the electrons.
