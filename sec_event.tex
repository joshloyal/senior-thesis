\section{Event Selection}
\label{sec:event}

Although the ATLAS detector measured billions of proton-proton collisions in 2011
and 2012 only a small fraction contain \HToZg decays. Therefore, an efficient
means of identifying these events using the tracking, particle identification, 
energy, and momentum information provided by the ATLAS detector is developed.
This section begins with a summary of the data samples used in this analysis
as well as any pre-analysis kinematic requirements applied to the data.
Next the algorithms used to reconstruct and identify the final state muons,
electrons, and photons from the crude information provided by the ATLAS detector
are outlined. Finally, the criteria used to interpolate back to the $Z$ and 
Higgs bosons is summarized. 

\subsection{Data Samples}
To ensure that the analyzed data is of high quality, events in which the ATLAS
detector is not fully operational, or shows data quality problems are excluded.
The resulting integrated luminosity after the trigger and data quality 
requirements corresponds to 20.7 \ifb (4.6 \ifb), with a relative uncertainty
of 3.6\% (1.8\%) at $\rts =$ 8 TeV ($\rts =$ 7 TeV) 
\cite{Aad:2011dr, Aad:2013ucp}.
Unless explicitly specified, the selection criteria
at $\rts = 8 \TeV$ and $\rts = 7 \TeV$ are identical. Values quoted in parentheses
correspond to the $\rts = 7 \TeV$ run period.


The data quality is determined by requiring the event to belong to a good runs list 
(GRL) generated by the ATLAS HSG2 group, and rejecting events with measurement 
errors in the lAr calorimeter, measurement errors in tile calorimeter, or are 
determined to be corrupt. 
The fraction of events removed in data by the GRL requirement
is around 7\% in 2011 data and 4-5\% in 2012 data. In addition, the fraction of
events with lAr, tile, or corrupt events is 0.3\% in 2011 and 0.5-0.6\% in 2012 
\cite{Abreu:1500529}.

In order to remain as unbiased as possible, the data is selected using the lowest
threshold, unprescaled single-lepton and di-lepton triggers. For the
single-muon trigger the transverse momentum, \pt, threshold is 24 (18) GeV, while
for the single-electron trigger the transverse energy, \et, threshold is 24 (20) GeV.
For the di-muon triggers, the thresholds are \pt $>$ 13 (10) GeV for each muon, while
for the di-electron triggers the thresholds are \et $>$ 12 GeV for each electron.
An asymmetric di-muon trigger is used at $\rts = 8 \TeV$ that requires one muon with 
a \pt $>$ 18 GeV and another with \pt $>$ 8 GeV. 
The trigger efficiency for signal
events passing the selection described is around 99\% for events in which the
$Z$ boson decays into $ee$ (the $ee\gamma$ channel) and 92\% for events where
the $Z$ boson decays to $\mu\mu$ pairs (the $\mu\mu\gamma$ channel). The lower
efficiency of the muon trigger is due to the reduced geometric acceptance of the muon
trigger system in the $|\eta| < 1.05$ region \cite{ATLAS-CONF-2013-009}.

In addition to an event trigger, the datasets are pre-filtered to 
reduce their size, allowing for a reasonable computing time.
The skims used were originally developed for data-driven photon identification
efficiency measurements using radiative $Z \to \ell\ell\gamma$ decays and
are based on loose kinematic and quality selections on the leptons. The
skims require at least one primary vertex with three associated tracks,
a lepton transverse momenta and photon transverse energy greater than 10 GeV,
{\tt loose++} electron identificaiton or {\tt loose} muon identification for
the lepton candidates, and at least one same flavor lepton pair with invariant mass
greater than 10 GeV. These filters retain roughly 3 - 4\% of the original events
\footnote{See this wiki for more details 
{\tt https://twiki.cern.ch/twiki/bin/viewauth/AtlasProtected/RadiativeZDAODs}.}.

\subsection{Primary Vertex Definition}
Events that pass the selection described below always have at least one
reconstructed primary vertex. These primary vertices are determined from fits
to tracks reconstructed in the inner detector and consistent with a common origin.
The interactions produced at the LHC are a result of collisions between the quarks
and gluons that make up the protons in the beams. These QCD processes are
classified as hard or soft depending on the amount of momentum transferred 
between the interacting particles. Of the roughly 20 proton-proton interactions
that occur during each bunch crossing, almost all of them are soft QCD processes.
This means there is very low momentum transfer and they mostly consist of glancing 
collisions between the quarks. However, the QCD process that produces a Higgs boson 
(gluon-gluon fusion) is a hard interaction with high momentum transfer.
In order to ensure that the primary vertex originated from the original hard 
interaction, the vertex is chosen with the largest sum of the 
squared transverse momenta of the tracks associated to it.

\subsection{Final State Particle Reconstruction and Selection}
Recall that two oppositely charged leptons ($\ell = e, \mu$) and a photon ($\gamma$)
make up the final state particles of a \HToZg event. This section
contains information regarding the reconstruction and identification of these 
particles from their energy deposition in the calorimeters and their tracking
information provided by the inner detector and the muon spectrometer.
In addition, the kinematic requirements enforced in order to select $\ell\ell\gamma$
triples originating from a Higgs decay are outlined. The sections proceed
as follows: muons, electrons, then photons.

\subsubsection*{Muon Reconstruction and Identification}
In ATLAS four kinds of muon candidates are distinguished depending on the
way they are reconstructed: stand-alone muons, combined muons, segment tagged muons,
and calorimeter tagged muons\cite{ATLAS-CONF-2010-064}.
\begin{itemize}
    \item \textbf{Stand-alone muons (SA):} SA muons are reconstructed entirely from
    tracking information provided by the muon spectrometer. The muons momentum
    is extrapolated back to the interaction point while taking into account the
    deviating effects from multiple coulomb scattering and energy loss in the 
    transversed material. SA muons utilize the extended coverage of the muon 
    spectrometer 
    and are used in the forward region ($2.5 < |\eta| < 2.7$) outside the 
    inner detector's coverage. Their use increases the overall acceptance 
    of the analysis. To increase the purity of these tracks, hits
    in all three muon spectrometer components where they are expected to pass
    are required.

    \item \textbf{Combined muons (CB):} The momentum information recorded by
    the MS is combined with the momentum measured in the inner detector. The
    muon trajectory in the inner detector also provides information about the 
    impact parameter of the muon trajectory with respect to the primary vertex.

    \item \textbf{Segment tagged muon (ST):} A trajectory in the inner detector
    is identified as a muon if the trajectory extrapolated to the muon spectrometer
    can be associated with straight track segments in the precision muon chambers,
    i.e the muon drift tubes or the cathode strip chambers.

    \item \textbf{Calorimeter tagged muon (CT):} A trajectory in the inner detector
    is identified as a muon if the associated energy deposition in the calorimeters
    is compatible with the hypothesis of a minimizing ionizing particle\footnote{
    A minimum ionizing particle (MIP) is defined as a particle whose rate 
    of mean energy loss through matter is close to the minimum. 
    The EM calorimeter is designed such that muons are MIPs, so this information
    can be used to identify muons in the calorimeter.}.
    CT muons are used to cover the region
    $|\eta| \approx 0$ which is not equipped with muon chambers. The purity
    of these muons is further increased by requiring their
    transverse momentum exceed 15 GeV.
\end{itemize}
In this analysis, CB, ST or SA muons identified with the STACO reconstruction 
algorithm as well as CT muons are used.

\subsubsection*{Muon Selection}
In the center of the barrel region ($|\eta| < 0.1$), which lacks muon spectrometer
coverage, calorimeter tagged muons with $\pt > 15 \GeV$ are used. All muon candidates
identified as muons using the information from the muon spectrometer (combined,
stand-alone, segment tagged) are required  to have a transverse momentum 
$\pt > 10 \GeV$. Each of these muons is used in the region where they have
the highest purity: combined and stand-alone muons are required to have $|\eta| < 2.7$
, and stand-alone muons are required to have $2.5 < |\eta| < 2.7$. The inner
detector tracks associated to muons that are identified inside the ID acceptance are
required to have a minimum number of associated hits in each of the ID sub-detectors
(to ensure good track reconstruction). A summary of the inner-detector hit 
requirements is summarized in \refT{tab:muonidhits}. In addition, the inner
detector tracks are required to have transverse (longitudinal) impact parameter
$d_0$ ($z_0$) with respect to the primary vertex, smaller than 1 mm (10 mm).

\begin{table}[hbt]
   \centering
   \begin{tabular}{cc}
      \hline
      \hline
      \multicolumn{2}{c}{2012} \\
      \hline
      ID Si hit requirement & expectBLayerHit=false or numberOfBLayerHits $\geq$ 1 \\
                            & No. of pixel hits $+$ No. of crossed dead pixel sensors $ > $ 0 \\
                            & No. of SCT hits $+$ No. of crossed dead SCT sensors $ > $ 4 \\
                            & No. of pixel holes $+$ No. of SCT holes $ < $ 3 \\
      TRT hit requirements: $0.1 < |\eta|\le1.9$& Hits + Outliers $>$ 5 \& $\frac{\rm{Outliers}}{\rm{Hits + outliers}}<0.9 $\\
TRT hit requirements: $|\eta| < 0.1 $ or $|\eta|\geq1.9$ & if (Hits + Outliers $>$ 5):  $\frac{\rm{Outliers}}{\rm{Hits + outliers}} <0.9 $\\
\hline
\hline
      \multicolumn{2}{c}{2011} \\
      \hline
      ID Si hit requirement & expectBLayerHit=false or numberOfBLayerHits $\geq$ 1 \\
                            & No. of pixel hits $+$ No. of crossed dead pixel sensors $ > $ 1 \\
                            & No. of SCT hits $+$ No. of crossed dead SCT sensors $ > $ 5 \\
                            & No. of pixel holes $+$ No. of SCT holes $ < $ 3 \\
      TRT hit requirements: $|\eta|<1.9$& Hits + Outliers $>$ 5 \& $\frac{\rm{Outliers}}{\rm{Hits + outliers}}<0.9 $\\
TRT hit requirements: $|\eta|\geq1.9$ & if (Hits + Outliers $>$ 5):  $\frac{\rm{Outliers}}{\rm{Hits + outliers}} <0.9 $\\
\hline
\hline
  \end{tabular}
   \caption{List of Inner Detector hit requirements for the muon
     tracks reconstructed in the ID.\label{tab:IDhits}}
   \label{tab:muonidhits}
\end{table}

\subsubsection*{Electron and Photon Reconstruction}
The electron and photon reconstruction are seeded from clusters of energy deposited
in the electromagnetic calorimeter. The reconstruction is designed to separate 
electrons, unconverted photons, and photons which have converted in the detector
material to electron-positron pairs. The clusters are matched to tracks and to 
conversion vertex candidates, which have 
been reconstructed in the inner detector and extrapolated to the second layer of
the calorimeter \cite{ATL-PHYS-PUB-2011-007}.
Electromagnetic clusters with a matching track in the inner-detector are classified
as electrons. The interaction of photons with the ATLAS detector can be
classified into two main categories: converted and unconverted photons. Converted
photons are classified by the presence of at least one ID track matching an
electromagnetic cluster originating from a vertex in the tracking volume. Clusters
without a matching track are classified as unconverted photons. Tracks matched
to electron candidates (and, for 8 TeV data, from photon conversions) and having
enough associated hits in the silicon detectors are fitted using a Gaussian-Sum
Filter, which accounts for bremsstrahlung energy loss \cite{ATLAS-CONF-2012-047}.
The energies of the clusters are calibrated, separately for electrons, converted
and unconverted photon candidates, to account for energy losses upstream of the 
calorimeter and for energy leakage outside of the cluster. In data, the calibration 
is refined by applying $\eta$-dependent correction factors, which are of the order
of $\pm 1\%$, determined from data-to-MC comparisons of the Z boson mass peak in
$\Zboson \to \epem$ events \cite{Aad:2011mk}. 
In addition, the energy measurement of converted
photons is improved with corrections, evaluated as a function of the conversion
radius (i.e. the distance from the beam-line at which the photon converts into an
\epem pair) based on dedicated Monte-Carlo simulation studies.

\subsubsection*{Electron Selection}
Electron candidates are required to have a transverse energy greater than 10 GeV,
pseudorapidity $|\eta| < 2.47$, and a well-reconstructed ID track pointing to an 
electromagnetic calorimeter cluster. The cluster should satisfy a set of 
identification criteria that require the longitudinal and transverse shower
profiles to be consistent with those expected for electromagnetic showers (
{\tt loose++} in ATLAS jargon) \cite{Aad:2011mk}. The electron four-momentum is formed
using the energy measured by the electromagnetic 
calorimeter and the track azimuth
and pseudorapidity measured in the inner detector. The electron track is required
to have a hit in the $b$-layer when passing through an active $b$-layer module
(this reduces fake electron candidates from photon conversions)
and is also required to have a longitudinal impact parameter, with respect to
the primary vertex, smaller than 10 mm.

\subsubsection*{Photon Selection}
Photon candidates are required to have a transverse energy greater than 15 GeV.
In order to exploit the fine segmentation of the first layer of the electromagnetic
calorimeter to discriminate between genuine prompt photons (photons directly produced
by parton-parton collisions) and fake photons with jets, the photon candidate 
pseudorapidity must satisfy $\abseta < 1.37$ or $1.52 < \abseta < 2.37$. The
photon four-momentum is formed using the energy measured by the calorimeter in the
direction determined by the nominal interaction point and the energy-weighted
barycenter of all cluster cells in the second layer of the calorimeter. Photons
reconstructed near regions of the calorimeter affected by read-out or high-voltage 
failures are not accepted. The identification of photons is performed through a
cut-based selection based on shower shapes measured in the first two longitudinal
layers of the electromagnetic calorimeter and on the leakage in the hadronic
calorimeter. To further suppress hadronic background, an isolation requirement is 
applied. The calorimeter isolation transverse energy $E_{T}^{\text{iso}}$ is
estimated by summing the transverse energy of three-dimensional positive-energy
topological clusters reconstructed in the electromagnetic and hadronic 
calorimeters in a cone of $\Delta R = 0.4$ around the photon candidate,
where the region within $0.125 \times 0.175$ in $\eta \times \phi$ around the
photon barycenter is excluded. The isolation energy is corrected by subtracting
the estimated contributions from leakage of the photon energy outside of the
excluded region and, using the technique described in 
Refs. \cite{Cacciari:2008gn, Cacciari:2009dp},
from the underlying event and additional $pp$ interactions. Photon candidates
are required to have an isolation transverse energy less than 4 GeV.

\subsubsection*{Overlap Removal}
An overlap removal between the electrons and muons that pass all selection
criteria and share the same inner detector track within a cone of $\Delta R < 0.02$
is performed: if the muon is identified by the MS, then the electron candidate is 
discarded, otherwise the muon candidate is rejected. Photon candidates that are
within $\Delta R < 0.3$ of a selected electron or muon candidate are also rejected,
thus suppressing background from final state radiation (FSR) $Z+\gamma$ events.

\subsection{$Z(\ell\ell)$ Reconstruction and Selection}
\subsection{$H(Z\gamma)$ Reconstruction and Selection}
