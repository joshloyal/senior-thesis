\section{Introduction} % (non-mathematical)
Elementary particle physics addresses the question, `What is the world made of?' 
at the most fundamental level, i.e. on the smallest scale of size. 
One could easily imagine a world where the elementary building blocks of 
nature come in a vast variety of shapes and sizes. Just think of the multitude of 
parts that work together to make up an ordinary car. Remarkably this is not
the case. The tiniest chunks of nature come in a small number of different types
(electrons, quarks, photons, and so on), which are replicated in astronomical
quantities to form our macroscopic world. This is not the whole story.
Not only is elementary particle physics concerned with identifying the various 
fundamental actors of nature, but also how these actors interact amongst each other.
It is during this investigation that the familiar forces of nature make their
appearance. 
Furthermore, a puzzling asymmetry exists amongst these particles and forces:
some forces are long range while others operate only on an atomic
scale. Linked to this phenomena is one of the most pressing questions in particle
physics over the past few decades: why do some fundamental particles have mass?
This paper presents a search for the Higgs boson, the particle 
behind mass generation in the Standard Model (SM), by probing its interactions with 
two other important actors: the photon and the $Z$ boson.

On July 4th, 2012 the ATLAS and CMS collaborations both announced the 
discovery of a particle consistent with the 
Standard Model Higgs boson~\cite{ATLAS_Higgs,CMS_Higgs}, indicating that
the discovery of the mechanism that gives the fundamental particles their mass
may be at hand. In addition, the ATLAS experiment measured 
this Higgs-like boson's mass to be close to 125 GeV~\cite{ATLAS_Higgs_Dec12}.
This measurement is only the beginning of a challenging program of `Higgs
identification' through which the consistency of this new boson with the SM
Higgs will be verified or disproved. For this reason it is now becoming
increasingly important to measure the properties of this new scalar particle
as well as its rate of decay for the largest number of experimentally
viable decay channels. These analyses could result in tension with the SM
Higgs prediction, for instance the rate of one or more measured decay channels
may differ from the SM prediction. 

The main decay modes probed in the July 4th searches were
the $H \to \gamma\gamma$ channel, the $H \to WW^* \to 2\ell2\nu$ channel 
and the 'golden channel', $H \to ZZ^* \to 4\ell$. 
However, little attention has been paid to the 
$H \to Z\gamma \to \ell^+\ell^-\gamma$ channel
despite the fact that its event rate is comparable to that of the golden  
channel for a 125 GeV Standard Model Higgs boson. The main reason for this is 
the low branching ratio for $Z \to \ell^+\ell^-$, the
probability that a $Z$ boson will decay into two leptons, 
makes the $Z\gamma$ channel statistically limited. 
However, there are a few important properties that make a study of the 
$Z\gamma$ channel compelling: 
1) all final state particles can be measured well with the ATLAS detector;  
2) the Higgs mass could be measured from the total invariant mass spectrum; 
3) the spin of the Higgs can be studied by analyzing the angular distribution 
of the decay products, and 
4) this channel can be used for setting limits on the Higgs coupling constants.
In addition, the simplest mechanism for probing new physics is through 
an enhancement/suppression of loop-induced decays\footnote{
A loop means that particles of any mass can instantly materialize and 
then disappear during the decay process. The brief-lived virtual particles
are usually $W$ bosons, but other particles with similar behavior can enter
the loop, including many beyond the Standard Model particles. This makes decays 
involving loops very sensitive to new physics at high masses.},
such as those mediating \HToZg. Also,
the ratio of $\gamma\gamma$ to $Z\gamma$ decay rates can
be used to discriminate between certain models of 
new physics~\cite{Zg_newPhy_1,Zg_newPhy_2, Zg_newPhy_3}.
Overall, the measurement of \HToZg is important for extending our understanding
of the Higgs mechanism.

This report documents the measurements of the \HToZg production rate observed
using data from $pp$ collisions provided by the LHC. In the following, the
theory behind the \HToZg decay is discussed in Section~\ref{sec:theory},
the ATLAS detector is described in Section~\ref{sec:experiment}, and the
criteria used to select \HToZg decays from the data is described in 
Section~\ref{sec:event}. A comparison between the selected data sample and
the theoretical simulation is presented in 
Section~\ref{sec:compare}. The discrimination between signal and background 
events is performed by means of an unbinned maximum likelihood fit.
The signal model used for the fit is described
in Section~\ref{sec:signal}, while the choice of the background model adopted
in the fit is motivated in Section~\ref{sec:background}. After a description of
the systematic uncertainties in Section~\ref{sec:sys}, Section~\ref{sec:results}
presents the results of the combined analysis of the 7 and 8 TeV datasets.
